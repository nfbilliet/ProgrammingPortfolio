\documentclass{book}

\begin{document}
    \section{Overview of the C++ language}
    language characteristics
    \begin{itemize}
        \item C++ is an object oriented language 
        \begin{itemize}
            \item Object Oriented Programming (OOP) is a programming model that organizes software around \textbf{data and objects}, i.e. structures that are reusable, rather than functions and logic (Functional programming)
            \item OOP follows four distinct principles
            \begin{enumerate}
                \item \textbf{Abstraction}, removal or generalization of physical, spatial or temporal details or attributes of specific systems. 
                \item \textbf{Inheritance}, implementation of a hierarchical structure where objects can be defined as a subclass of superclass. The subclass is said to inherit or share the characteristics and functions of the superclass while also incorporating unique and new characteristics and functions (akin to the parent-child relationship)
                \item \textbf{Encapsulation}, describes bundling data and methods that work on that data within one unit, like a class. It allows you to hide specific information and control access to the object’s internal state
                \item \textbf{Polymorhpism}, refers to the ability of a programming language to interpret objects in different ways based on their class or data type. In essence, it is the ability of a single method to be applied to derived classes and achieve a proper output.
            \end{enumerate}
        \end{itemize}
    \end{itemize}
    \section{Classes}
    \subsection{Overloading operators}
    
\end{document}